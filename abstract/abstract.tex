
\section*{Abstract}

The Alps are among the most affected regions in the world by climate change, displaying some of the strongest glacier retreat rates. Long-term interactions between society, mountain ecosystems and glaciers in the region raise important questions on the future evolution of glaciers and their derived environmental and socio-economical  impacts. In order to correctly assess the regional response of glaciers in the French Alps to climate change, there is a need for adequate modelling tools. In this work, we explore new ways to tackle both glacier evolution and glacio-hydrological modelling at a regional scale. Glacier evolution modelling has traditionally been performed using empirical or physical approaches, which are becoming increasingly challenging to optimize with the ever growing amount of available data. Here, we present, to our knowledge, the first effort ever to apply deep learning (i.e. deep artificial neural networks) to simulate the evolution of glaciers. Since both the climate and glacier systems are highly nonlinear, traditional linear mass balance models offer a limited representation of climate-glacier relationships. We show how important nonlinearities in glacier mass balance are captured by deep learning, substantially improving model performance over linear methods. 

This novel method was first applied in a study to reconstruct annual mass balance changes for all glaciers in the French Alps for the 1967-2015 period. Using climate reanalyses, topographical data and glacier inventories, we demonstrate how such an approach can be successfully used to reconstruct large-scale mass balance changes from observations. This study also offered new insights on how glaciers evolved in the French Alps during the last half century, confirming the rather neutral observed mass balance rates in the 1980s and displaying a well-marked acceleration in mass loss from the 2000s on-wards. Important differences between regions are found, with the Mont-Blanc massif presenting the lowest mass loss and the Chablais being the most affected one. Secondly, we applied this modelling framework to simulate the future evolution of all glaciers in the region under multiple (N=29) climate change scenarios. Our estimates indicate that most ice volume in the region will be lost by the end of the 21st century whatever the future climate scenarios. We predict average glacier volume losses of 75\%, 80\% and 88\% under RCP 2.6 (n=3), RCP 4.5 (n=13) and RCP 8.5 (n=13), respectively. By the end of the 21$^{st}$ century the French Alps will be largely ice-free, with glaciers only remaining in the Mont-Blanc and Pelvoux massifs. Our analyses indicate that high-altitude accumulation basins are the most decisive factor determining the future survival of glaciers in the French Alps, followed by higher latitudes. With the warming climate, glaciers retreat to higher elevations in an effort to regain equilibrium with the present climate. Glaciers in high-altitude massifs have a larger altitudinal span to retreat to, offering a colder climate to survive the higher temperatures. In doing so, shrinking glaciers induce major changes in their experienced climate signal, with a reduction in temperature of up to 400 positive degree days (PDD) per year and an increase in snowfall of up to 230 mm per year. We further demonstrate how these complex climate-glacier relationships are highly nonlinear, with linear glacier mass balance models overestimating extreme positive and negative mass balance rates. By taking into account these nonlinear interactions in our model, a largely negative bias in long-term glacier projections was prevented thanks to the nonlinear properties of deep learning. 

This marked glacier retreat in the French Alps will produce an array of consequences that will impact water resources during the warmest months of the year. Glaciers provide cold fresh water resources well after all snow has melted during summer, essential to inhabitants in the region that depend on it for agriculture, industry, ski resorts, hydropower generation and domestic uses. Moreover, several aquatic and terrestrial ecosystems depend on these late summer water resources, that keep water temperature low and ecosystems humid throughout the year. Predicting these changes is of paramount importance in order to correctly anticipate the resulting impacts and to design adequate mitigation strategies. Current hydrological models used in France generally suffer from a simplified representation of glaciers, modelling them as static ice reservoirs. This representation is highly problematic in the current context of rapid glacier retreat. Here, we introduce an updated dynamic representation of glaciers for the J2K hydrological model, validated in a case study in the Arvan partially glacierized catchment. By taking into account the daily area evolution of glaciers, this process-based hydrological model represents an excellent tool to assess the hydrological consequences of glacier retreat at the scale of the French Alps. 
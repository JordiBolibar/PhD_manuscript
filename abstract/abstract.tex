
\section*{Abstract}

The European Alps are among the most affected regions in the world by climate change, displaying some of the strongest glacier retreat rates. Long-term interactions between society, mountain ecosystems and glaciers in the region raise important questions on the future evolution of glaciers and their derived environmental and socioeconomical impacts. In order to correctly assess the regional response of glaciers in the French Alps to climate change, there is a need for adequate modelling tools. In this work, we explore new ways to tackle both glacier evolution and glacio-hydrological modelling at a regional scale. Glacier evolution modelling has traditionally been performed using empirical or physical approaches, which are becoming increasingly challenging to optimize with the ever growing amount of available data. Here, we present, to our knowledge, the first effort ever to apply deep learning (i.e. deep artificial neural networks) to simulate the evolution of glaciers. Since both the climate and glacier systems are highly nonlinear, traditional linear mass balance models offer a limited representation of climate-glacier interactions. We show how important nonlinearities in glacier mass balance are captured by deep learning, substantially improving model performance over linear methods. 

This novel method was first applied in a study to reconstruct annual mass balance changes for all glaciers in the French Alps for the 1967-2015 period. Using climate reanalyses, topographical data and glacier inventories, we demonstrate how such an approach can be successfully used to reconstruct large-scale mass balance changes from observations. This study also offered new insights on how glaciers evolved in the French Alps during the last half century, confirming the rather neutral observed mass balance rates in the 1980s and displaying a well-marked acceleration in mass loss from the 2000s onwards. Important differences between regions are found, with the Mont-Blanc massif presenting the lowest mass loss and the Chablais being the most affected one. Secondly, we applied this modelling framework to simulate the future evolution of all glaciers in the region under multiple (N=29) climate change scenarios. Our estimates indicate that most ice volume in the region will be lost by the end of the 21$^{st}$ century independently from future climate scenarios. We predict average glacier volume losses of 74\%, 80\% and 88\% under RCP 2.6 (n=3), RCP 4.5 (n=13) and RCP 8.5 (n=13), respectively. By the end of the 21$^{st}$ century the French Alps will be largely ice-free, with glaciers only remaining in the Mont-Blanc and Pelvoux massifs. From this point, we use these results as a case study to investigate the effects of nonlinear glacier response to climatic forcing. We show that linear glacier MB models ignore nonlinearities in glacier MB compared to nonlinear deep learning, overestimating extreme positive and negative MB rates. We demonstrate that these patterns are also found in empirical temperature-index linear models, which are used by the vast majority of global and regional glacier evolution models. Depending on future climate scenarios, this behaviour can introduce a significant bias in glacier MB projections, reaching up to -20\% by the end of the century. This could therefore have remarkable consequences on projections of future glacier evolution, suggesting that current global glacier models based on linear MB relationships might be giving estimates of future sea-level rise that are too high for climate scenarios with the highest greenhouse gases emissions, but also too low for climate scenarios with drastic reductions in emissions. 

This marked glacier retreat in the French Alps will produce an array of consequences that will impact water resources during the warmest months of the year. Glaciers provide cold fresh water resources well after all snow has melted during summer, essential to inhabitants in the region that depend on it for agriculture, industry, ski resorts, hydropower generation and domestic use. Moreover, several aquatic and terrestrial ecosystems depend on these late summer water resources, that keep water temperature low and ecosystems humid throughout the year. Predicting these changes is of paramount importance in order to correctly anticipate the resulting impacts and to design adequate mitigation strategies. Hydrological models currently used in France generally suffer from a simplified representation of glaciers, modelling them as static ice reservoirs. This representation is highly problematic in the current context of rapid glacier retreat. Here, we introduce a dynamic representation of glaciers for the process-based J2K hydrological model, validated in a case study in the Arvan partially glacierized catchment. By taking into account the daily area evolution of glaciers, this updated glacio-hydrological model represents an excellent tool to assess the diverse hydrological consequences of glacier retreat at the scale of the French Alps. 
\\

\textbf{Keywords}: French Alps, glaciers, machine learning, deep learning, modelling, hydrology

\newpage

\section*{Résumé}

Les Alpes européennes sont parmi les régions du monde les plus touchées par le changement climatique, avec des taux de recul des glaciers parmi les plus élevés. Les interactions à long terme entre la société, les écosystèmes de montagne et les glaciers de la région soulèvent d'importantes questions sur l'évolution future des glaciers et les impacts environnementaux et socio-économiques qui en découlent. Afin d'évaluer correctement la réponse régionale des glaciers des Alpes françaises au changement climatique, il est nécessaire de disposer d'outils de modélisation adéquats. Dans ce travail, nous explorons de nouvelles façons d'aborder à la fois l'évolution des glaciers et la modélisation glacio-hydrologique à l'échelle régionale. La modélisation de l'évolution des glaciers a traditionnellement été réalisée à l'aide d'approches empiriques ou physiques, dont l'optimisation est de plus en plus difficile compte tenu de la quantité croissante de données disponibles. Ici, nous présentons, à notre connaissance, le premier effort jamais entrepris pour appliquer l'apprentissage profond (i.e. des réseaux neuronaux artificiels profonds) pour simuler l'évolution des glaciers. Comme les systèmes climatique et glaciaire sont tous deux fortement non linéaires, les modèles traditionnels linéaire de bilan de masse offrent une représentation limitée des interactions entre le climat et les glaciers. Nous montrons comment des non-linéarités importantes liées au bilan de masse des glaciers sont capturées par une méthode d'apprentissage profond, ce qui améliore considérablement les performances des modèles par rapport aux méthodes linéaires. 

Cette nouvelle méthode a été appliquée pour la première fois dans une étude visant à reconstruire les changements annuels du bilan de masse de tous les glaciers des Alpes françaises pour la période 1967-2015. En utilisant des réanalyses climatiques, des données topographiques et des inventaires de glaciers, nous démontrons comment une telle approche peut être utilisée avec succès pour reconstruire les changements de bilan de masse à grande échelle à partir d'observations. Cette étude a également apporté de nouveaux éclairages sur l'évolution des glaciers dans les Alpes françaises au cours du dernier demi-siècle, confirmant les taux de bilan de masse observés plutôt neutres dans les années 1980 et montrant une accélération bien marquée de la perte de masse à partir des années 2000. On constate des différences importantes entre les régions, le massif du Mont-Blanc présentant la perte de masse la plus faible et le Chablais étant le plus touché. Ensuite, nous avons appliqué ce cadre de modélisation pour simuler l'évolution future de tous les glaciers de la région selon de multiples scénarios de changement climatique (N=29). Nos estimations indiquent que la plupart du volume de glace dans la région sera perdue d'ici la fin du XXIe siècle, indépendamment des scénarios climatiques futurs. Nous prévoyons des pertes moyennes de volume des glaciers de 74\%, 80\% et 88\% dans le cadre des scénarios RCP 2.6 (n=3), RCP 4.5 (n=13) et RCP 8.5 (n=13), respectivement. D'ici la fin du XXIe siècle, les Alpes françaises seront en grande partie libres de glace, avec des glaciers ne subsistant que dans les massifs du Mont-Blanc et du Pelvoux. Nous avons ensuite utilisé ces résultats comme un cas d'étude pour analyser les effets de la réponse non linéaire des glaciers au forçage climatique. Nous montrons que les modèles linéaires de bilan de masse de glaciers ignorent systématiquement les non-linéarités dans le bilan de masse par rapport à l'apprentissage profond non linéaire, en surestimant les taux positifs et négatifs extrêmes du bilan de masse. Nous démontrons que ces limitations se retrouvent également dans les modèles linéaires empiriques de type dégré-jour, qui sont utilisés par la grande majorité des modèles d'évolution des glaciers au niveau mondial et régional. En fonction des scénarios climatiques futurs, ce comportement peut introduire un biais significatif dans les projections de bilan de masse des glaciers, pouvant atteindre -20\% d'ici la fin du siècle. Cela pourrait donc avoir des conséquences remarquables sur les projections de l'évolution future des glaciers, ce qui suggère que les modèles globaux actuels des glaciers basés sur des relations linéaires de bilan de masse pourraient donner des estimations de l'élévation future du niveau des mers qui sont trop élevées pour les scénarios climatiques avec les plus fortes émissions de gaz à effet de serre, mais aussi trop faibles pour les scénarios climatiques avec des réductions drastiques des émissions. 

Ce recul marqué des glaciers dans les Alpes françaises aura un ensemble de conséquences avec notamment un impact sur les ressources en eau pendant les mois les plus chauds de l'année. Les glaciers fournissent des ressources en eau douce froide bien après la fonte des neiges en été, ce qui est essentiel pour les habitants de la région qui en dépendent pour l'agriculture, l'industrie, les stations de ski, la production d'énergie hydroélectrique et l'utilisation domestique. En outre, plusieurs écosystèmes aquatiques et terrestres dépendent de ces ressources en eau de fin d'été, qui maintiennent la température de l'eau à un niveau bas et l'humidité des écosystèmes tout au long de l'année. La prévision de ces changements est d'une importance capitale pour anticiper correctement les impacts qui en résulteront et pour concevoir des stratégies d'atténuation adéquates. Les modèles hydrologiques actuellement utilisés en France souffrent généralement d'une représentation simplifiée des glaciers, les modélisant comme des réservoirs de glace statiques. Cette représentation est très problématique dans le contexte actuel de recul rapide des glaciers. Nous présentons ici une représentation dynamique actualisée des glaciers pour le modèle hydrologique J2K, validée dans une étude de cas dans le bassin versant partiellement englacé de l'Arvan. En prenant en compte l'évolution quotidienne de la surface des glaciers, ce modèle hydrologique basé sur les processus représente un excellent outil pour évaluer les conséquences hydrologiques du recul des glaciers à l'échelle des Alpes françaises. \\

\textbf{Mots clés}: Alpes françaises, glaciers, apprentissage automatique, apprentissage profond, modélisation, hydrologie
\chapter{Introduction}
\label{chap:intro}

\begin{flushright}
\textit{The most important thing about expedition food is that there is some.}\\
Eric Shipton
\end{flushright}

\section*{Why are we concerned about glaciers?}
\emph{"Changes in the longer-lived components of the cryosphere (e.g., glaciers) are the result of an integrated response to climate, and the cryosphere is often referred to as a 'natural thermometer'. But as our understanding of the complexity of this response has grown, it is increasingly clear that elements of the cryosphere should rather be considered as a 'natural climate-meter', responsive not only to temperature but also to other climate variables (e.g., precipitation). However, it remains the case that the conspicuous and widespread nature of changes in the cryosphere (in particular, sea ice, glaciers and ice sheets) means these changes are frequently used emblems of the impact of changing climate."} This citation, quoted from the fifth Intergovernmental Panel on Climate Change (IPCC) assessment report \citep{vaughan_2013}, highlights the gap between the striking observations of massive retreat of glaciers and the in-depth understanding of the causes of these changes. 


As a first step, good observations of glacier mass changes are needed to make a reliable diagnostic of the past and recent glacier mass changes. The glacier  contribution to the current sea level rise (SLR) for the beginning of the twenty-first century was only recently comprehensively assessed \citep{gardner_reconciled_2013}, and the future evolution of glaciers and ice sheets is the main source of uncertainty in the SLR projections \citep{church_sea_2013}. For the beginning of the twenty-first century, and more precisely for the 2003-2009 period, glaciers contributed to 30 \% of the observed SLR \citep{gardner_reconciled_2013}. Then, the second step is to develop models of glacier evolution, which can help to understand the processes responsible for glacier changes, and for example attribute the share of anthropic forcings \citep{marzeion_attribution_2014}. They can predict the future evolution of glaciers under different climate scenarios \citep{marzeion_past_2012,radic_regional_2014,huss_new_2015,marzeion_limited_2018}. Third, the glacier evolution models can be further used as inputs for hydrological models to assess the impacts of glacier changes on water resources for populations living downstream at a local scale \citep{huss_toward_2017,huss_global-scale_2018,milner_glacier_2017}.

However, the lack of good observational data disrupts the implementation of glacier models, this is the case for High Mountain Asia (HMA) glaciers \citep{bolch_state_2012,azam_review_2018}.

\section*{Why in High Mountain Asia?}
The lack of glacier measurements, despite extensive glacier coverage, is all the more problematic, as HMA glaciers sustain the river discharge during the dry months for some densely populated basins \citep{kaser_contribution_2010,immerzeel_climate_2010,schaner_contribution_2012,huss_toward_2017}, and therefore, realistic projections of HMA glacier changes are crucially needed. Satellite based techniques can partially alleviate the lack of field studies, but they are limited to pluri-annual averages \citep{kaab_contrasting_2012,gardelle_region-wide_2013}.


Under a similar CO$_2$ emission scenario (Representative Concentration Pathway (RCP) 4.5), different glacier models predict a mass loss of 49 to 55 \% of the current glacier mass for the entire HMA by 2100 \citep{marzeion_past_2012,radic_regional_2014,huss_new_2015,kraaijenbrink_impact_2017}.
The good agreement among these models for the end of the century is surprising, because they are calibrated with different strategies, and they strongly differ for the early twenty-first century mass changes. For instance, \citet{marzeion_past_2012} model prediction for 2000-2016 is more than twice as negative as \citet{gardner_reconciled_2013} observation (for the period 2003-2009) on which \citet{huss_new_2015} model is calibrated. This example rises important questions about the models calibration and about the relevance of the processes modeled.

The large tongues of HMA glaciers are often covered by a thick layer of debris \citep{scherler_spatially_2011}, which effect was included in only one of the above mentioned models \citep{kraaijenbrink_impact_2017}.

\section*{Why a focus on debris?}
At the scale of HMA, the model prediction of \citet{kraaijenbrink_impact_2017} does not significantly differ from the other models, despite an explicit modeling of the debris effect on ice ablation. However, at the scale of the local Dudh Koshi catchment, two studies found irreconcilable results for the future of glaciers by 2100 \citep{rowan_modelling_2015,shea_modelling_2015}. Different choices in glacier modeling led to the prediction of glacier mass reduction of 8-10 \% in one case \citep{rowan_modelling_2015} and 84-95 \% in the other case \citep{shea_modelling_2015}. The two studies are not directly comparable, because they investigated different areas and used different climate change inputs, but the main source of discrepancy is the modeling of the debris effect on ablation and the modeling of debris transport in one case \citep{rowan_modelling_2015}.

Glaciological knowledge is based mostly on debris free glaciers, but the extent of the debris cover is expected to increase in a context of global warming, with a widespread slowdown of glacier tongues \citep{heid_repeat_2012}, which favors debris emergence \citep{kirkbride_formation_2013,anderson_modeling_2016,rowan_modelling_2015,wirbel_modelling_2018}. The recent increase in debris cover extent has been documented, for example, in the Alps \citep[e.g.,][]{deline_change_2005,gardent_multitemporal_2014}, in Garhwal \citep[e.g.,][]{bhambri_glacier_2011} and in the Everest region \citep[e.g.,][]{thakuri_tracing_2014}.\\



Consequently, within the course of the coming years, we might partially change our vision of glacier tongues, and debris. It was before considered as anecdotal feature, but has become very common. The potential influence of debris on the glacier evolution is still unclear, and it is therefore needed to better understand the relationship between debris and glacier mass balance. Within this long-term prospects, the aim of this PhD work is to assess the recent evolution of HMA glaciers and to quantify the influence of debris on the glacier mass balance. This work is based on a multi-scale approach where large scale observations help to build a statistical intuition and/or validate models behavior, while, in parallel, fine scale approaches are developed to study processes, even if they are localized and their conclusions are not easy to extrapolate.

\section*{A short note to the reader}
This manuscript organization follows a general direction from large scales to small scales. It starts with a review of the current state of the art knowledge about HMA climate and glaciers (chapter \ref{chap:glaciers_and_climate}), at the end of which the detailed research questions addressed in this manuscript can be found. The main body consists in three chapters, each of which is based on an article (one published, one accepted and one in review). The articles are introduced by a short note and for some of them I present further development and research directions. A conclusion summarizes this work and provides future research directions (chapter \ref{chap:conclu}). An article published in 2016 and based on a work I did for my master's thesis is appended at the end of the manuscript, as it was an important basis for the chapter \ref{chap:paper_cliff}.

This structure implies some repetitions among the different chapters and a couple of inconsistencies, such as the use of \mwe or m w.e. yr$^{-1}$ for the mass balance units, which are imposed by the different journal styles \citep[both in compliance with][]{cogley_glossary_2011}.
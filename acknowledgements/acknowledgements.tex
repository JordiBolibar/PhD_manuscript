\section*{Acknowledgements}

This PhD work has been, by all means, a collective effort. I would have never been able to accomplish this without the help, kind words or time from a multitude of people. I am extremely grateful for that, for this project has been an incredible human experience. There are a lot of people that I would like to thank, and I sincerely hope I will not forget anyone. First of all, I would like to thank my parents, for giving me the opportunity and freedom to pursue my studies, which gave me the independence to control my professional career and aim it towards the things I love and matter to me. All of this would not have been possible without Kadia, my closest companion, the hidden co-author, who took care of me during these years despite the long rants on glaciers, climate and machine learning. I am also extremely grateful for the family I have. Laia, Mariona, Andrew, for all the time shared together, especially in New Zealand; and my grandparents, especially avi Robert, who taught me maths and physics, and who I wish I could show this work. 

Then, I would like to thank my supervisors, who gave me the possibility to do this work, and who always respected my vision and way of working. Antoine, for his availability, dedicating whatever time needed to my problems, and protecting me from several ordeals regarding French paperwork. Being able to discuss in Spanish at the beginning of the PhD was a great way to engage in deep discussions on glaciers, and improved my confidence as a total newcomer to the field. Isabelle, who guided me in the most kind way, and provided invaluable insights on climatology and hydrology. I particularly appreciate the freedom I was given to work on the ideas that drive me, knowing the cost that this represented on her research part. For that, I am grateful, but also sorry for not having managed to include them in a better way. Eric and Thomas, who provided insightful comments throughout my work, and whose hydrological expertise helped me during the last part of this work. Special thanks go to Clovis Galiez, who has played the role of a bonus supervisor. I learnt so much about machine learning from him, and our discussions have been to me some of the most stimulating ones during these three years, including several times where his suggestions unlocked problems that I had for weeks. Sven Kralisch has also provided me with guidance on hydrological modelling during the last months. Despite finally not being able to visit Jena, I have always been surprised by his sincere kindness, availability and will to help. 

My PhD companions have proved to be my best allies for such a long journey, with many people I would like to thank. Julien, Joseph, Gabi and Nathan shared with me some of the most memorable adventures on two wheels, from the Alps to Utah. Maria, Gabi, both Juliens, Jai, Jonathan, Joseph and Ambroise shared many hours of skinning up and skiing down snowy mountains, reminding us why we love them so much. Ugo, for the long discussions in the office, attempting to solve the world's problems besides finishing our PhDs, and for the amazing time in Stiappa. Lucas, Olivier and Juan Pedro, for being excellent office partners, and for the stimulating conversations. Romain, Marion, Astrid, Diego, Hans, Foteini, Albane, Sammy, Jinwha, Laura, Pedro, Peter, Sarah, Claudio, Maxim, Fanny and Marco, for the great times shared in the lab, in conferences, in the mountains or at the bar. I acknowledge the long term help and support from Molts Ànims, my good old friends in Sant Cugat, who have always been there and who always make me feel home despite the long seven years living abroad. 

I am also grateful to Delphine, Olivier and Bruno, who taught me how to measure glacier mass balance in the field, in some of the most breathtaking landscapes I have ever seen. I had so much fun doing that, but I will particularly remember the day when, together with Delphine, we measured all the ablation stakes in Mer de Glace in one long and arduous push. I am especially thankful to the members of my PhD jury for accepting to review this work. I have been lucky enough to have exactly the jury I wished for, and I really appreciate the time and effort that it involves. Moreover, I would also like to thank Christian, Thomas and Agnès, for taking part in my thesis committee, and guiding me with insightful comments  and making it feel more like a conversation than an evaluation. There are many people in the lab(s) that I would also like to thank, for the friendly and interesting conversations in the cafeteria and for the shared small talk: Jérémie, Nico, Sophie, Patrick, Jean-Manu, Gilles, Martin, Lionel and Vincent. 

Finally, I am grateful for the many scientific interactions I had outside the lab, particularly during conferences and paper reviews. I would like to thank Fabien Maussion, Ben Marzeion and Matthias Huss, who besides being an inspiration to me, provided constructive feedback that truly managed to improve my work. I can only hope to have the same luck with reviewers in the future. I am also very thankful to Daniel Farinotti, who always showed kind interest in my work from the beginning, and who is currently helping me to pursue my scientific ideas for a postdoc. I had the chance to meet Fernando Pérez in San Francisco last year, resulting in stimulating discussions on scientific modelling. I hope we will be able to collaborate in the future, once the currently troubled times calm down. I have very fond memories of my winter school in Mendoza, Argentina. I learnt a lot from Lucas Ruiz, Pierre Pitte, Lidia Ferri, Laura Zalazar, Maxi Viale and Mariano Masiokas, and I met some amazing people with whom I experienced the beauty of the Andes. I also had the pleasure to be part of the NASA ICESat-2 Hackweek, which despite having a virtual format, allowed me to learn from, interact and collaborate with great people from all around the world. At last, I am also thankful for the discussions and feedback from Harry Zekollari for my latest paper, with whom I share very similar research interests and the pleasure of writing long e-mails with lots of figures.

I acknowledge the funding of my PhD work by INRAE, the Labex OSUG@2020 (Investissement d'Avenir, ANR ANR10 LABX56), the Auvergne-Rhône-Alpes region through the BERGER project, the VIP Mont-Blanc project (ANR-14 CE03-0006-03) and the CNES (KALEIDOS-Alpes and ISIS projects grants).

I like to believe that these three years have been a small taste of what science is about: a diverse, inclusive, open and collective discussion on the investigation and discovery of nature. 

\bigskip
\bigskip
\bigskip
\bigskip

\begin{flushright}
\begin{small}
\textit{This manuscript has been written and assembled in the beautiful small village of Stiappa, Tuscany (Italy).}
\end{small}
\end{flushright}
